\documentclass[10pt,a4paper]{jhwhw}
\usepackage[utf8]{inputenc}
%Paquetes Necesarios
\usepackage{amsmath}
\usepackage{amsfonts}
\usepackage{amssymb}
\usepackage{makeidx}
\usepackage[spanish,es-lcroman]{babel}
\usepackage{titling}
\usepackage{amsthm}
\usepackage{enumerate}
\usepackage{tikz}
\usepackage{latexsym}
\usepackage{cite}
\usepackage{titlesec}
\usepackage{fancybox}
\usepackage{xparse}
\usepackage{longdivision}
\usepackage{xcolor}
%Quitar el identado de todos los parrafos
\setlength{\parindent}{0cm}
%Para agregar el identado en cada item de enumerate o cualquier otro, usar [\hspace{1cm}(a)]

%Comandos de Letras
\newcommand{\R}{\mathbb{R}}
\newcommand{\N}{\mathbb{N}}
\newcommand{\Z}{\mathbb{Z}}
\newcommand{\Q}{\mathbb{Q}}
\newcommand{\C}{\mathbb{C}}

%Informacion del del autor del libro y localizacion
\author{Resuelto por: \href{https://www.facebook.com/ruller}{Raúl García}\\Pagina Web: \href{https://rull3r.github.io/}{MateTips}\\Correo: rull3r@hotmail.com}
\date{Venezuela\\ \today \\}
\title{Solucionario \\\href{https://www.amazon.com/-/es/James-Kraft/dp/1138063479}{An Introduction to Number Theory with Cryptography\\James Kraft, Lawrence Washington}\\}
%Para el indice alfabetico
\makeindex

%Marca de agua en el documento
\usepackage{draftwatermark}
\SetWatermarkText{\textsc{\href{https://rull3r.github.io/}{Visitame en MateTips}}} % por defecto Draft 
\SetWatermarkScale{1} % para que cubra toda la página
%\SetWatermarkColor[rgb]{1,0,0} % por defecto gris claro
\SetWatermarkAngle{55} % respecto a la horizontal

\longdivisionkeys{style = german}
\begin{document}
	
\problema{ }\label{pro:1}
	Muestre que:
	\begin{enumerate}[\hspace{1cm}(a)]
		\item $11\nmid 27$.
		\item $15\nmid 9$.
		\item $12\nmid 44$.
		\item $7\nmid 90$.
	\end{enumerate}
	\solution 
	\part
	Haciendo un procedimiento clásico de división:\\
	\begin{center}
		\intlongdivision[german division sign = $\,\div\,$]{27}{11}
	\end{center}
	Obtenemos que el resto es $5$ por lo tanto $11$ no divide a $27$\QEPD
	\part
	Haciendo un procedimiento clásico de división:\\
	\begin{center}
		\intlongdivision[german division sign = $\,\div\,$]{9}{15}
	\end{center}
	Obtenemos que el resto es $9$ por lo tanto $15$ no divide a $9$\QEPD
	\part
	Haciendo un procedimiento clásico de división:\\
	\begin{center}
		\intlongdivision[german division sign = $\,\div\,$]{44}{12}
	\end{center}
	Obtenemos que el resto es $8$ por lo tanto $12$ no divide a $44$\QEPD
	\part
	Haciendo un procedimiento clásico de división:\\
	\begin{center}
		\intlongdivision[german division sign = $\,\div\,$]{90}{7}
	\end{center}
	Obtenemos que el resto es $6$ por lo tanto $7$ no divide a $90$\QEPD
\end{document}